%\documentclass[overlap,line]{cv}
\documentclass[overlap,line]{cv}
\newsectionwidth{0.2in}

\oddsidemargin -.5in
\evensidemargin -.5in
\textwidth=6.1in
\itemsep=0in
\parsep=0in
% if using pdflatex:
%\setlength{\pdfpagewidth}{\paperwidth}
%\setlength{\pdfpageheight}{\paperheight} 
\setlength{\parskip}{0.05in}

\newenvironment{list1}{
  \begin{list}{\ding{113}}{%
      \setlength{\itemsep}{0in}
      \setlength{\parsep}{0in} \setlength{\parskip}{0in}
      \setlength{\topsep}{0in} \setlength{\partopsep}{0in} 
      \setlength{\leftmargin}{0.17in}}}{\end{list}}
\newenvironment{list2}{
  \begin{list}{$\bullet$}{%
      \setlength{\itemsep}{0in}
      \setlength{\parsep}{0in} \setlength{\parskip}{0in}
      \setlength{\topsep}{0in} \setlength{\partopsep}{0in} 
      \setlength{\leftmargin}{0.2in}}}{\end{list}}

\pagenumbering{gobble}

\begin{document}

\name{Andrey Asadchev \vspace*{.1in}}

\begin{resume}
\section{\sc Contact Information}
\vspace{.05in}
(515) 337-5362 \\
asadchev@gmail.com

\section{\sc Summary}
\vspace{.05in}
Computational scientist passionate about C++, high-performance computing, and low-level code optimization.
Interests include C++ template expressions and meta-programming, parallel linear algebra,
computational chemistry, and machine learning. \\
US permanent resident (green card holder).\\
Fluent in English, native Russian speaker, conversational level of French.


\section{\sc Education}
\vspace{.05in}
{\bf PhD, Physical Chemistry}, Dec 2012 \\
Iowa State University, Ames, IA
\vspace*{.02in}
\begin{list2}
%% \item Advisor:  Mark S. Gordon
\item Emphasis on computational quantum chemistry.
\item Dissertation: {\it Modernizing the core quantum chemistry algorithms}, \\
 {http://lib.dr.iastate.edu/etd/12915/}
\end{list2}

{\bf BSc, Computer Science},  May, 2005\\
Valdosta State University, Valdosta, GA
\vspace*{.02in}
\begin{list2}
\item Magna Cum Laude
\end{list2}


\section{\sc Technical Skills}
\vspace{.05in}
\begin{tabular}{ l @{\hskip 1em} l }
Programming Languages: & C++/C++11, Python, C, FORTRAN 77/90 \\
C++ Concepts: & Object oriented design, metaprogramming, expression templates \\
C++ Libraries: & Standard Template Library (STL), Boost (including MPL and Fusion)\\
Machine Learning: & Tensorflow, Scikit, R, Skytree \\
Scientific Computing: & BLAS, LAPACK, Eigen, Trillions, Mathematica, Matlab/Octave \\
Parallel Programming: & MPI, CUDA, TBB, OpenMP, ARMCI \\
Parallel I/O: & HDF5, NetCDF, Lustre, PVFS \\
Code optimization: & SSE and AVX vectorization with intrinsics/assembly \\
Performance analysis: & Linux perf, Intel VTune/Parallel Studio, PAPI \\
Debugging: & gdb, Valgrind, DDT, TotalView \\
Build tools: & CMake, autotools \\
Collaboration: & git, Atlassian stack (JIRA, Stash), Agile \\
HPC Environments: & IBM BlueGene, Cray XT, OpenMPI, MPICH, PBS/Torque \\
UNIX/Linux: & POSIX API programming, bash shell scripting, Linux administration \\
\end{tabular}


\section{\sc Work and Research Experience}
\vspace{.05in}
{\bf Infosys Limited}, Palo Alto, CA \\
{\it Senior Product Technical Architect} \hfill {\bf 2017 - Present}

Development and integration of machine learning features in Infosys products
\vspace{.05in}
\begin{list2}
\item Integrated Tensorflow into Infosys ML offerings.
\item Introduced C++/Python interoperability via Boost.Python to allow for rapid development and prototyping.
\item Continuous enhancement of core code base and build system.
\end{list2}

\newpage

{\bf Skytree, Inc.}, San Jose, CA \\
{\it HPC C++ Developer} \hfill {\bf 2013 - 2017}

Participated in the development of the Skytree C++ Machine Learning (ML) product.
\vspace{.05in}
\begin{list2}
\item Refactored the C++ codebase according to modern C++ patterns; \\
  reduced the number of lines of code, improved maintainability and compilation speed.
\item Identified and resolved key computational and memory bottlenecks; \\
  significantly improved overall execution time and reduced memory footprint.
\item Implemented core dense and sparse linear algebra objects and kernels.
\item Designed and implemented several distributed ML algorithms using OpenMP and MPI.
\item Worked with data scientists to address feature requests and resolve bugs.
\end{list2}

{\bf Virginia Tech}, Blacksburg, VA \\
{\it PostDoctoral Researcher} \hfill {\bf 2012 - 2013}

Participated in the development of Massively Parallel Quantum Chemistry (MPQC) package.
\vspace{.05in}
\begin{list2}
\item Modernized the MPQC C++ code base by using Boost and Eigen linear algebra library.
\item Ported MPQC to CMake; simplified build and deployment across various HPC platforms.
\item Developed fully distributed OpenMP/MPI sparse Davidson eigensolver
  with out-of-core capabilities, scalable to 1000s of cores.
\item Conducted pilot research to optimize key computational kernels on SSE/AVX processors.
\end{list2}


{\bf Iowa State University}, Ames, IA \\
{\it Research Assistant} \hfill {\bf 2008 - 2012}

Developed computational chemistry C++ library (libcchem) with GPU capabilities
and FORTRAN bindings.
\vspace{.05in}
\begin{list2}
\item Implemented automatically generated vectorized C++ Rys Quadrature algorithm; \\
  30-40\% faster than the original FORTRAN code.
\item Implemented scalable multi-threaded Hartree-Fock algorithm with constant memory overhead.
\item Implemented Rys Quadrature and Hartree-Fock algorithms on GPU using CUDA C++; \\
  showed speedups on the order of 10-17 times.\\
  In the process resolved several issues in Boost C++ libraries due to CUDA compiler.
\item Designed and implemented distributed in-core and out-of-core
  data storage suitable for matrix and tensor computations with C++
  object oriented interface and integration with BLAS.
\item Implemented scalable distributed perturbation theory and Coupled Cluster
  algorithms with low memory footprint and ability to utilize GPU devices via CuBLAS. \\
  The implementation showed orders of magnitude reduction in memory use, as
  well as improvement in speed and overall scalability (1000s of cores),
  over the original FORTRAN code.
\end{list2}


{\bf Ames Laboratory}, Ames, IA \\
{\it Research Assistant} \hfill {\bf 2006 - 2008}

Participated in the development of The General Atomic and Molecular Electronic Structure System
(GAMESS).
\vspace{.05in}
\begin{list2}
\item Added ARMCI remote memory implementation to GAMESS.
\item Implemented execution script to run GAMESS in various MPI (OpenMPI, MPICH, IMPI)
  and batch system environments (LSF, PBS, Torque, LoadLeveler).
\item Ported GAMESS to Blue Gene and Cray XT.
\item Replaced FORTRAN code with BLAS/LAPACK calls, parallelized linear algebra routines; \\
  improved overall program performance, allowed GAMESS to scale to 1000s of cores.
\end{list2}

\newpage

\section{\sc Teaching Experience}
\vspace{.05in}
{\bf Virginia Tech}, Blacksburg, VA \\
{\em (SICM)2 Summer School} \hfill {\bf 2013}

\begin{list2}
\item Taught GPU/CUDA programming workshop to computational sciences graduate students.
\end{list2}

{\bf Iowa State University}, Ames, IA \\
{\em Chemistry Instructor}  \hfill {\bf 2005 - 2006}

\begin{list2}
\item Taught general chemistry lab and recitation sections. 
\end{list2}


\section{\sc Publications and Presentations}
\vspace{.05in}

{\it A Fast and Flexible Coupled Cluster Approach} \\
A. Asadchev, M. S. Gordon \\
Journal of Chemical Theory and Computation 2013 9(8)

{\it New Multithreaded Hybrid CPU/GPU Approach to Hartree–Fock} \\
A. Asadchev, M. S. Gordon \\
Journal of Chemical Theory and Computation 2012 8 (11)

{\it Uncontracted Rys Quadrature Implementation of up to G Functions on graphical processing units (GPUs)} \\
A. Asadchev, V. Allada, J. Felder, B. M. Bode, M. S. Gordon, T. L. Windus \\
Journal of Chemical Theory and Computation 2010 6(3)

{\it Accelerating Quantum Chemistry Research using GPUs} \\
A. Asadchev, J. Felder \\
GPU Technology Conference, NVIDIA, San Jose, 2009

%% {\it Performance of Electronic Structure Calculations on BG/L and XT4 Computers} \\
%% A. Asadchev, B. M. Bode, M. S. Gordon \\
%% Journal of Computational and Theoretical Nanoscience, 2009 6(6)

{\it Uncontracted Rys Quadrature on GPU} \\
A. Asadchev \\
Path to Petascale, NCSA, 2009

{\it Performance of Electronic Structure Calculations on Blue Gene/L and Cray XT4} \\
A. Asadchev \\
IEEE/ACM Supercomputing 2008 Poster

\end{resume}
\end{document}




